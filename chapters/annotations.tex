
\chapter{Annotations}\doublelabel{annotations}

Annotations are intended for storing extra information about a model,
such as graphics, documentation or versioning, etc. A Modelica tool is
free to define and use other annotations, in addition to those defined
here, according to \autoref{vendor-specific-annotations}. The only requirement is that any tool
shall save files with all annotations from this chapter and all
vendor-specific annotations intact. To ensure this, annotations must be
represented with constructs according to the Modelica grammar (for
replaceable class declarations with a constraining-clause also refer to
\autoref{constraining-clause-annotations}). The specification in this document defines the
semantic meaning if a tool implements any of these annotations.

\section{Vendor-Specific Annotations}\doublelabel{vendor-specific-annotations}

A vendor may -- anywhere inside an annotation -- add specific, possibly
undocumented, annotations which are not intended to be interpreted by
other tools. Two variants of vendor-specific annotations exist; one
simple and one hierarchical. Double underscore concatenated with a
vendor name as initial characters of the identifier are used to identify
vendor-specific annotations.

\begin{example}
\begin{lstlisting}[language=modelica]
annotation (
   Icon(coordinateSystem(extent={{-100,-100}, {100,100}}),
        graphics={__NameOfVendor(Circle(center={0,0}, radius=10))}) );
\end{lstlisting}
This introduces a new graphical primitive \lstinline!Circle! using the
hierarchical variant of vendor-specific annotations.
\begin{lstlisting}[language=modelica]
annotation (
   Icon(coordinateSystem(extent={{-100,-100}, {100,100}}),
        graphics={Rectangle(extent={{-5,-5},{7,7}}, __NameOfVendor_shadow=2)}) );
\end{lstlisting}
This introduces a new attribute \lstinline!__NameOfVendor_shadow!
for the \lstinline!Rectangle! primitive using the simple variant of
vendor-specific annotations.
\end{example}

\section{Annotations for Documentation}\doublelabel{annotations-for-documentation}

\begin{lstlisting}[language=grammar]
documentation-annotation:
   annotation "(" Documentation "(" "info" "=" STRING ["," "revisions" "=" STRING ] ")" ")"
\end{lstlisting}
The \lstinline!Documentation! annotation can contain the \lstinline!info! annotation
giving a textual description, the \lstinline!revisions! annotation giving a list
of revisions and other annotations defined by a tool {[}\emph{The
\lstinline!revisions! documentation may be omitted in printed documentation}{]}.
How the tool interprets the information in \lstinline!Documentation! is
unspecified. Within a string of the \lstinline!Documentation! annotation, the
tags \lstinline!<HTML>! and \lstinline!</HTML>! or
\lstinline!<html>! and \lstinline!</html>! define
optionally begin and end of content that is HTML encoded. For external
links see \autoref{external-resources}. Links to Modelica classes may be defined with
the HTML link command using scheme \lstinline!Modelica!, e.g.,
\begin{lstlisting}[language=modelica]
<a href="Modelica://MultiBody.Tutorial">MultiBody.Tutorial</a>
\end{lstlisting}

Together with scheme \lstinline!Modelica! the (URI) fragment specifiers
\lstinline!#diagram!, \lstinline!#info!, \lstinline!#text!, \lstinline!#icon! may be used to reference different
layers. Example:
\begin{lstlisting}[language=modelica]
<a href="Modelica://MultiBody.Joints.Revolute#info">Revolute</a>
\end{lstlisting}
\begin{lstlisting}[language=grammar]
preferred-view-annotation:
   annotation "(" preferredView "=" ("info" | "diagram" | "text") ")"
\end{lstlisting}

The \lstinline!preferredView! annotation defines the default view when selecting the
class. \lstinline!info! means info layer, i.e., the documentation of the class,
\lstinline!diagram! means diagram layer and \lstinline!text! means the Modelica text layer.
\begin{lstlisting}[language=grammar]
documentation-class-annotation:
   annotation "(" DocumentationClass "=" true ")"
\end{lstlisting}

Only allowed as class annotation on any kind of class and implies that
this class and all classes within it are treated as having the
annotation \lstinline!preferredView="info"!. If the annotation \lstinline!preferredView! is
explicitly set for a class, it has precedence over a \lstinline!DocumentationClass!
annotation \emph{{[}A tool may display such classes in special ways. For
example, the description texts of the classes might be displayed instead
of the class names, and if no icon is defined, a special information
default icon may be displayed in the package browser.{]}}

\section{Annotations for Code Generation}\doublelabel{annotations-for-code-generation}
\begin{lstlisting}[language=grammar]
code-annotation:
   annotation"(" codeGenerationFlag "=" ( false | true ) ")"

codeGenerationFlag :
   "Evaluate" | "HideResult" | "Inline" | "LateInline" | "GenerateEvents"
\end{lstlisting}
These annotations can influence the code generation. The details are
defined in the next table:

\begin{longtable}[]{|p{4.2cm}|p{10cm}|}
\hline \endhead
\lstinline!Evaluate!&
The annotation Evaluate can occur in the component declaration, its type
declaration, or a base-class of the type-declaration. In the case of
multiple conflicting annotations it is handled similarly to modifiers
(e.g., an Evaluate-annotation on the component declaration takes
precedence). The annotation Evaluate only has effect for a component
declared with the prefix \lstinline!parameter!.

If \lstinline!Evaluate = true!, the model developer proposes to utilize the value
for the symbolic processing. In that case, it is not possible to change
the parameter value after symbolic pre-processing.

If \lstinline!Evaluate = false!, the model developer proposes to not utilize the
value of the corresponding parameter for the symbolic processing.

{[}\lstinline!Evaluate! \emph{is for example used for axis of rotation parameters in
the} \lstinline!Modelica.Mechanics.MultiBody! \emph{library in order to improve the
efficiency of the generated code}{]}\\ \hline
\lstinline!HideResult! &
\lstinline!HideResult = true! defines that the model developer proposes to not show
the simulator results of the corresponding component {[}\emph{e.g., it
will not be possible to plot this variable}{]}.

\lstinline!HideResult = false! defines that the developer proposes to show the
corresponding component {[}\emph{if a variable is declared in a
protected section, a tool might not include it in a simulation result.
By setting} HideResult = false\emph{, the modeler would like to have the
variable in the simulation result, even if in the protected section}{]}.

{[}\lstinline!HideResult! \emph{is for example used in the connectors of the}
\lstinline!Modelica.StateGraph! \emph{library to not show variables to the modeler
that are of no interest to him and would confuse him}{]}\\ \hline
\lstinline!Inline! &
Has only an effect within a function declaration.

If \lstinline!Inline = true!, the model developer proposes to inline the
function. This means, that the body of the function is included at all
places where the function is called.

If \lstinline!Inline = true!, the model developer proposes to not inline the
function.

% Added "The annotation" to avoid bad formatting.
{[}The annotation \lstinline!Inline = true! \emph{is for example used in}
Modelica\allowbreak{}.Mechanics\allowbreak{}.MultiBody\allowbreak{}.Frames \emph{and in functions of}
Modelica\allowbreak{}.Media \emph{to have no overhead for function calls such as
resolving a vector in a different coordinate system and at the same time
the function can be analytically differentiated, e.g., for index
reduction needed for mechanical systems.}{]}\\ \hline
\lstinline!LateInline!
&
Has only an effect within a function declaration.

If \lstinline!LateInline = true!, the model developer proposes to inline the
function after all symbolic transformations have been performed
\emph{{[}especially differentiation and inversion of functions; for
efficiency reasons it is then useful to replace all function calls with
identical input arguments by one function call, before the inlining{]}}.

If \lstinline!LateInline = false!, the model developer proposes to not inline
the function after symbolic transformations have been performed.

\lstinline!Inline=true, LateInline=false! is identical to \lstinline!Inline=true!.

\lstinline!Inline=true, LateInline=true! is identical to \lstinline!LateInline=true!.

\lstinline!Inline=false, LateInline=true! is identical to \lstinline!LateInline=true!.

{[}\emph{This annotation is for example used in
Modelica\allowbreak{}.Media\allowbreak{}.}Water\allowbreak{}.IF97\_Utilities\allowbreak{}.T\_props\_ph \emph{to provide in
combination with common subexpression elimination the automatic caching
of function calls. Furthermore, it is used in order that a tool is able
to propagate specific enthalpy over connectors in the Modelica\_Fluid
library.}{]}\\ \hline
\lstinline!InlineAfterIndexReduction!\strut
&
Has only an effect within a function declaration.

If \lstinline!true!, the model developer proposes to inline the function after the
function is differentiated for index reduction, and before any other
symbolic transformations are performed. This annotation cannot be
combined with annotations \lstinline!Inline! and \lstinline!LateInline!.
\\ \hline
\lstinline!GenerateEvents!\strut
&
Has only an effect within a function declaration

If \lstinline!GenerateEvents = true!, the model developer proposes that crossing
functions in the function should generate events (one possibility of
doing this is to inline the function and generate events for the inlined
function).

{[}\emph{This annotation is for example used in}
Modelica\allowbreak{}.Media\allowbreak{}.Water\allowbreak{}.IF97\_Utilities.phase\_dT \emph{to indicate that
the output should generate an event when it changes.}{]}\\ \hline

\end{longtable}
\begin{lstlisting}[language=grammar]
smoothOrder-annotation:
   annotation"(" smoothOrder "=" UNSIGNED-NUMBER ")" |
   annotation"(" smoothOrder "(" normallyConstant "=" NAME
  ["," normallyConstant "=" NAME] ")"
  "=" UNSIGNED-NUMBER ")"
\end{lstlisting}
This annotation has only an effect within a function declaration.
\lstinline!smoothOrder! defines the number of differentiations of the function, in
order that all of the differentiated outputs are continuous provided all
input arguments and their derivatives up to order \lstinline!smoothOrder! are
continuous {[}\emph{This means that the function is at least}
C\textsuperscript{smoothOrder}. \lstinline!smoothOrder = 1! \emph{means that the
function can be differentiated at least once in order that all output
arguments are still continuous, provided the input arguments are
continuous. If a tool needs the derivative of a function, e.g. for index
reduction or to compute an analytic Jacobian, the function can be
differentiated analytically at least smoothOrder times}{]}.

The optional argument \lstinline!normallyConstant! of \lstinline!smoothOrder! defines that the
function argument \lstinline!NAME! is usually constant {[}\emph{A tool might check
whether the actual argument to} \lstinline!NAME! \emph{is a parameter expression at the
place where the function is called. If this is the case, the derivative
of the function might be constructed under the assumption that the
corresponding argument is constant, to enhance efficiency. Typically, a
tool would generate at most two different derivative functions of a
function: One, under the assumption that all} \lstinline!normallyConstant! \emph{arguments
are actually constant. And one, under the assumption that all input
arguments are time varying. Based on the actual arguments of the
function call either of the two derivative functions is used.}

\emph{This annotation is used by many functions of the} \lstinline!Modelica.Fluid!
\emph{library, such as}\\
\lstinline!Modelica.Fluid.Dissipation.PressureLoss.StraightPipe.dp_laminar_DP!\emph{,
since geometric arguments to these functions are usually constant}{]}

\section{Annotations for Simulation Experiments}\doublelabel{annotations-for-simulation-experiments}
\begin{lstlisting}[language=grammar]
experiment-annotation:
   annotation "(" "experiment" [ "(" [experimentOption
      {"," experimentOption}] ")" ] ")"

experimentOption:
  StartTime "=" [ "+" | "-" ] UNSIGNED-NUMBER |
  StopTime  "=" [ "+"  | "-"] UNSIGNED-NUMBER |
  Interval "=" UNSIGNED-NUMBER |
  Tolerance "=" UNSIGNED-NUMBER
\end{lstlisting}

The experiment annotation defines the default start time (StartTime) in
  {[}s{]}, the default stop time (StopTime) in {[}s{]}, the suitable time
  resolution for the result grid (Interval) in {[}s{]}, and the default
relative integration tolerance (Tolerance) for simulation experiments to
be carried out with the model or block at hand. If StartTime is not specified it is assumed to be \lstinline!0.0!.

\section{Annotation for single use of class}\doublelabel{annotation-for-single-use-of-class}

For state machines it is useful to have single instances of local
classes. This can be done using:
\begin{lstlisting}[language=modelica]
annotation(singleInstance=true)
\end{lstlisting}

The annotation singleInstance in a class indicates that there should
only be one component instance of the class, and it should be in the
same scope as the class is defined. The intent is to remove the class
when the component is removed and to prevent duplication of the
component.

\section{Annotations for Graphical Objects}\doublelabel{annotations-for-graphical-objects}

A graphical representation of a class consists of two abstraction
layers, icon layer and diagram layer showing graphical objects,
component icons, connectors and connection lines. The icon
representation typically visualizes the component by hiding hierarchical
details. The hierarchical decomposition is described in the diagram
layer showing icons of subcomponents and connections between these.

Graphical annotations described in this chapter ties into the Modelica
grammar as follows.
\begin{lstlisting}[language=grammar]
graphical-annotations :
  annotation "(" [ layer-annotations ] ")"

layer-annotations :
  ( icon\_layer | diagram\_layer ) [ "," layer-annotations ]
\end{lstlisting}
Layer descriptions (start of syntactic description):
\begin{lstlisting}[language=grammar]
icon-layer :
  "Icon" "(" [ coordsys-specification "," ] graphics ")"

diagram-layer :
  "Diagram" "(" [ coordsys-specification "," ] graphics ")"
\end{lstlisting}
{[}\emph{Example}:
\begin{lstlisting}[language=modelica]
annotation (
   Icon(coordinateSystem(extent={{-100,-100}, {100,100}}),
        graphics={Rectangle(extent={{-100,-100}, {100,100}}),
                  Text(extent={{-100,-100}, {100,100}}, textString="Icon")}));
\end{lstlisting}
{]}

The graphics is specified as an ordered sequence of graphical
primitives, which are described below. First base-class contents is
drawn according to the order of the extends-clauses, and then graphical
primitives are drawn according to the order such that later objects can
cover earlier ones. {[}\emph{Note that the ordered sequence is
syntactically a valid Modelica annotation, although there is no
mechanism for defining an array of heterogeneous objects in
Modelica.}{]}

These Icon, Diagram, and Documentation annotations are only allowed
directly in classes (e.g. not on components or connections). The allowed
annotations for a short class definition is the union of the allowed
annotations in classes and on extends-clauses.

\subsection{Common Definitions}\doublelabel{common-definitions}

The following common definitions are used to define graphical
annotations in the later sections.

\begin{lstlisting}[language=modelica]
  type DrawingUnit = Real(final unit="mm");
  type Point = DrawingUnit[2] "{x, y}";
  type Extent = Point[2] "Defines a rectangular area {{x1, y1}, {x2, y2}}";
\end{lstlisting}
The interpretation of \lstinline!unit! is with respect to printer output in
natural size (not zoomed).

All graphical entities have a visible attribute which indicates if the
entity should be shown.

\begin{lstlisting}[language=modelica]
partial record GraphicItem
  Boolean visible = true;
  Point origin = {0, 0};
  Real rotation(quantity="angle", unit="deg")=0;
end GraphicItem;
\end{lstlisting}
The \lstinline!origin! attribute specifies the origin of the graphical item in the
coordinate system of the layer in which it is defined. The origin is
used to define the geometric information of the item and for all
transformations applied to the item. All geometric information is given
relative the \lstinline!origin! attribute, which by default is \{0, 0\}.

The \lstinline!rotation! attribute specifies the rotation of the graphical item
counter-clockwise around the point defined by the \lstinline!origin! attribute.

\subsubsection{Coordinate Systems}\doublelabel{coordinate-systems}

Each of the layers has its own coordinate system. A coordinate system is
defined by the coordinates of two points, the left (x1) lower (y1)
corner and the right (x2) upper (y2) corner, where the coordinates of
the first point shall be less than the coordinates of the second point
{[}\emph{a first quadrant coordinate system}{]}.

The attribute \lstinline!preserveAspectRatio! specifies a hint for the shape of
components of the class, but does not actually influence the rendering of the component.
If \lstinline!preserveAspectRatio! is true, changing the
extent of components should preserve the current aspect ratio of the coordinate
system of the class.

The attribute \lstinline!initialScale! specifies the default component size as
\lstinline!initialScale! times the size of the coordinate system of the class. An
application may use a different default value of \lstinline!initialScale!.

The attribute \lstinline!grid! specifies the spacing between grid points which can
be used by tools for alignment of points in the coordinate system
{[}\emph{e.g. ``snap-to-grid''}{]}. Its use and default value is
tool-dependent.

\begin{lstlisting}[language=modelica]
record CoordinateSystem
  Extent extent;
  Boolean preserveAspectRatio=true;
  Real initialScale = 0.1;
  DrawingUnit grid[2];
end CoordinateSystem;
\end{lstlisting}
{[}\emph{Example}: \emph{A coordinate system for an icon could for
example be defined as:}
\begin{lstlisting}[language=modelica]
CoordinateSystem(extent = {{-10, -10}, {10, 10}});
\end{lstlisting}
\emph{i.e. a coordinate system with width 20 units and height 20
units.}{]}

The coordinate systems for the icon and diagram layers are by default
defined as follows; where the array of \lstinline!GraphicsItem! represents an
ordered list of graphical primitives.

\begin{lstlisting}[language=modelica]
record Icon "Representation of the icon layer"
  CoordinateSystem coordinateSystem(extent = {{-100, -100}, {100, 100}});
  GraphicItem[:] graphics;
end Icon;

record Diagram "Representation of the diagram layer"
  CoordinateSystem coordinateSystem(extent = {{-100, -100}, {100, 100}});
  GraphicItem[:] graphics;
end Diagram;
\end{lstlisting}
The coordinate system (including preserveAspectRatio) of a class is
defined by the following priority:

\begin{enumerate}
\item
  The coordinate system annotation given in the class (if specified).
\item
  The coordinate systems of the first base-class where the extent on the
  extends-clause specifies a null-region (if any). Note that null-region
  is the default for base-classes, see \autoref{extends-clause}.
\item
  The default coordinate system CoordinateSystem(extent=\{\{-100,
  -100\}, \{100, 100\}\}).
\end{enumerate}

\subsubsection{Graphical Properties}\doublelabel{graphical-properties}

Properties of graphical objects and connection lines are described using
the following attribute types.

\begin{lstlisting}[language=modelica]
  type Color = Integer[3](min=0, max=255) "RGB representation";

  constant Color Black = zeros(3);
  type LinePattern = enumeration(None, Solid, Dash, Dot, DashDot, DashDotDot);
  type FillPattern = enumeration(None, Solid, Horizontal, Vertical,
  Cross, Forward, Backward, CrossDiag, HorizontalCylinder, VerticalCylinder, Sphere);
  type BorderPattern = enumeration(None, Raised, Sunken, Engraved);
  type Smooth = enumeration(None, Bezier);
  type EllipseClosure = enumeration(None, Chord, Radial);
\end{lstlisting}
The \lstinline!LinePattern! attribute \lstinline!Solid! indicates a normal line, \lstinline!None! an
invisible line, and the other attributes various forms of dashed/dotted
lines.

The \lstinline!FillPattern! attributes \lstinline!Horizontal!, \lstinline!Vertical!,
\lstinline!Cross!, \lstinline!Forward!,
\lstinline!Backward! and \lstinline!CrossDiag! specify fill patterns drawn with the line color
over the fill color.

The attributes \lstinline!HorizontalCylinder!, \lstinline!VerticalCylinder! and \lstinline!Sphere! specify
gradients that represent a horizontal cylinder, a vertical cylinder and
a sphere, respectively. The gradient goes from line color to fill color.

\includegraphics[width=2.08333in,height=1.66667in]{bezierpoints}

The border pattern attributes \lstinline!Raised!, \lstinline!Sunken! and \lstinline!Engraved! represent frames
which are rendered in a tool-dependent way -- inside the extent of the
filled shape.

The \lstinline!smooth! attribute specifies that a line can be drawn as straight line
segments (\lstinline!None!) or using a spline (\lstinline!Bezier!), where the line's points
specify control points of a quadratic Bezier curve.

For lines with only two points, the \lstinline!smooth! attribute has no effect.

For lines with three or more points (P\textsubscript{1},
P\textsubscript{2}, \ldots{}, P\textsubscript{n}), the middle point of
each line segment (P\textsubscript{12}, P\textsubscript{23}, \ldots{},
P\textsubscript{(n-1)n}) becomes the starting point and ending points of
each quadratic Bezier curve. For each quadratic Bezier curve, the common
point of the two line segment becomes the control point. For instance,
point P\textsubscript{2} becomes the control point for the Bezier curve
starting at P\textsubscript{12} and ending at P\textsubscript{23}. A
straight line is drawn between the starting point of the line and the
starting point of the first quadratic Bezier curve, as well as between
the ending point of the line and the ending point of the last quadratic
Bezier curve.

In the illustration above, the square points (P\textsubscript{1},
P\textsubscript{2}, P\textsubscript{3,} and P\textsubscript{4})
represent the points that define the line, and the circle points
(P\textsubscript{12}, P\textsubscript{23}, and P\textsubscript{34}) are
the calculated middle points of each line segment. Points
P\textsubscript{12}, P\textsubscript{2}, and P\textsubscript{23} define
the first quadratic Bezier curve, and the points P\textsubscript{23},
P\textsubscript{3}, and P\textsubscript{34} define the second quadratic
Bezier curve. Finally a straight line is drawn between points
P\textsubscript{1} and P\textsubscript{12} as well as between
P\textsubscript{34} and P\textsubscript{4}.

The values of the \lstinline!EllipseClosure! enumeration specify if and how the
endpoints of an elliptical arc are to be joined (see \autoref{ellipse} Ellipse).

\begin{lstlisting}[language=modelica]
  type Arrow = enumeration(None, Open, Filled, Half);
  type TextStyle = enumeration(Bold, Italic, UnderLine);
  type TextAlignment = enumeration(Left, Center, Right);
\end{lstlisting}
Filled shapes have the following attributes for the border and interior.

\begin{lstlisting}[language=modelica]
record FilledShape "Style attributes for filled shapes"
  Color lineColor = Black "Color of border line";
  Color fillColor = Black "Interior fill color";
  LinePattern pattern = LinePattern.Solid "Border line pattern";
  FillPattern fillPattern = FillPattern.None "Interior fill pattern";
  DrawingUnit lineThickness = 0.25 "Line thickness";
end FilledShape;
\end{lstlisting}
The extent/points of the filled shape describe the theoretical
zero-thickness filled shape, and the actual rendered border is then half
inside and half outside the extent.

\subsection{Component Instance}\doublelabel{component-instance}

A component instance can be placed within a diagram or icon layer. It
has an annotation with a \lstinline!Placement! modifier to describe the placement.
Placements are defined in term of coordinate systems transformations:

\begin{lstlisting}[language=modelica]
record Transformation
  Point origin = {0, 0};
  Extent extent;
  Real rotation(quantity="angle", unit="deg")=0;
end Transformation;
\end{lstlisting}
The origin attribute defines the position of the component in the
coordinate system of the enclosing class. The \lstinline!extent! defines the
position, size and flipping of the component, relative to the \lstinline!origin!
attribute. The \lstinline!extent! is defined relative to the \lstinline!origin! attribute of the
component instance. Given an extent \{\{x1, y1\}, \{x2, y2\}\},
x2\textless{}x1 defines horizontal flipping and y2\textless{}y1 defines
vertical flipping around the center of the object.

The \lstinline!rotation! attribute specifies rotation of the extent around the point
defined by the \lstinline!origin! attribute.

The graphical operations are applied in the order: scaling, flipping and
rotation.

\begin{lstlisting}[language=modelica]
record Placement
  Boolean visible = true;
  Transformation transformation "Placement in the diagram layer";

  Boolean iconVisible "Visible in icon layer; for public connector";
  Transformation iconTransformation "Placement in the icon layer; for public connector";
end Placement;
\end{lstlisting}
If no \lstinline!iconTransformation! is given the \lstinline!transformation! is also used for
placement in the icon layer. If no \lstinline!iconVisible! is given for a public connector the
\lstinline!visible! is also used for visibility in the icon layer.

{[}\emph{A connector can be shown in both an icon layer and a diagram
layer of a class. Since the coordinate systems typically are different,
placement information needs to be given using two different coordinate
systems. More flexibility than just using scaling and translation is
needed since the abstraction views might need different visual placement
of the connectors. The attribute} \lstinline!transformation! \emph{gives the placement in
the diagram layer and} \lstinline!iconTransformation! \emph{gives the placement in the icon
layer. When a connector is shown in a diagram layer, its diagram layer
is shown to facilitate opening up a hierarchical connector to allow
connections to its internal subconnectors.}{]}

For connectors, the icon layer is used to represent a connector when it
is shown in the icon layer of the enclosing model. The diagram layer of
the connector is used to represent it when shown in the diagram layer of
the enclosing model. Protected connectors are only shown in the diagram
layer. Public connectors are shown in both the diagram layer and the
icon layer. Non-connector components are only shown in the diagram
layer.

\subsection{Extends clause}\doublelabel{extends-clause}

Each extends-clause (and short-class-definition, as stated in \autoref{annotations-for-graphical-objects})
may have layer specific annotations which describe
the rendering of the base class' icon and diagram layers in the
subclass.

\begin{lstlisting}[language=modelica]
record IconMap
  Extent extent = {{0, 0}, {0, 0}};
  Boolean primitivesVisible = true;
end IconMap;

record DiagramMap
  Extent extent = {{0, 0}, {0, 0}};
  Boolean primitivesVisible = true;
end DiagramMap;
\end{lstlisting}
All graphical objects are by default inherited from a base class. If the
\lstinline!primitivesVisible! attribute is false, components and connections are
visible but graphical primitives are not.

\begin{itemize}
\item
  If the extent of the extends-clause defines a null region (the
  default), the base class contents is mapped to the same coordinates in
  the derived class, and the coordinate system (including
  preserveAspectRatio) can be inherited as described in
  \autoref{coordinate-systems}.
\item
  If the extent of the extends-clause defines a non-null region, the
  base class coordinate system is mapped to the region specified by the
  attribute extent, if preserveAspectRatio is true for the base class
  the mapping shall preserve the aspect ratio. The base class coordinate
  system (and preserveAspectRatio) is not inherited.
\end{itemize}

{[}\emph{Example}:

\begin{lstlisting}[language=modelica]
model A
  extends B annotation(
  IconMap(extent={{-100,-100}, {100,100}},primitivesVisible=false),
  DiagramMap(extent={{-50,-50}, {0,0}},primitivesVisible=true)
  );
end A;

model B
  extends C annotation(DiagramMap(primitivesVisible=false));
  ...
end B;
\end{lstlisting}
\emph{In this example the diagram of A contains the graphical primitives
from A and B (but not from C since they were hidden in B) -- the ones
from B are rescaled, and the icon of A contains the graphical primitives
from A (but neither from B nor from C).}

{]}

\subsection{Connections}\doublelabel{connections1}

A connection is specified with an annotation containing a \lstinline!Line! primitive
and optionally a Text-primitive, as specified below. {[}\emph{Example:}
\begin{lstlisting}[language=modelica]
  connect(a.x, b.x)
   annotation(Line(points={{-25,30}, {10,30}, {10, -20}, {40,-20}}));
\end{lstlisting}

{]}

The optional Text-primitive defines a text that will be written on the
connection line. It has the following definition (\emph{it is not equal
to the Text-primitive as part of graphics -- the differences are marked as bold lines}):
% NOTE: Technically just the names - not the entire lines are marked in bold
\begin{lstlisting}[language=modelica,escapechar=!,emph={horizontalAlignment,string,index}, emphstyle=\textbf]
record Text
  extends GraphicItem;
  extends FilledShape;
  Extent extent;
  String string;
  Real fontSize = 0 "unit pt";
  String fontName;
  TextStyle textStyle[:];
  Color textColor=lineColor;
  TextAlignment horizontalAlignment = if index<0 then TextAlignment.Right else TextAligment.Left;
  Integer index;
end Text;
\end{lstlisting}

The \lstinline!index! is one of the points of Line (numbered 1, 2, 3, \ldots{} where
negative numbers count from the end, thus -1 indicate the last one). The \lstinline!string!
may use the special
symbols \lstinline!"\%first"! and \lstinline!"\%second"! to indicate the connectors in the
connect-equation.

The \lstinline!extent! and \lstinline!rotation! are relative to the \lstinline!origin! (default \lstinline!{0,0}!)
and the \lstinline!origin! is relative to the point on the Line.

The textColor attribute defines the color of the text. The text is drawn
with transparent background and no border around the text (and without
outline). The contents inherited from FilledShape is deprecated, but kept for compatibility reasons.
The default value for \lstinline!horizontalAlignment! is deprecated.
Having a zero size for the extent is deprecated and is handled as if upper part is moved up an appropriate amount.

{[}\emph{Example:}
\begin{lstlisting}[language=modelica]
  connect(controlBus.axisControlBus1, axis1.axisControlBus) annotation (
      Text(string="%first", index=-1, extent=[-6,3; -6,7]),
      Line(points={{-80,-10},{-80,-14.5},{-79,-14.5},{-79,-17},{-65,-17},{-65,-65},
           {-25,-65}}));
\end{lstlisting}
\emph{Draws a connection line and adds the text \emph{axisControlBus1}
ending at \{-6, 3\}+\{-25, -65\} and 4 vertical units of space for the text.
Using a height of zero, such as \lstinline!extent=[-6,3; -6,3]! is deprecated, but gives similar result.}
{]}

\subsection{Graphical primitives}\doublelabel{graphical-primitives}

This section describes the graphical primitives that can be used to
define the graphical objects in an annotation.

\subsubsection{Line}\doublelabel{line}

A line is specified as follows:

\begin{lstlisting}[language=modelica]
record Line
  extends GraphicItem;
  Point points[:];
  Color color = Black;
  LinePattern pattern = LinePattern.Solid;
  DrawingUnit thickness = 0.25;
  Arrow arrow[2] = {Arrow.None, Arrow.None} "{start arrow, end arrow}";
  DrawingUnit arrowSize=3;
  Smooth smooth = Smooth.None "Spline";
end Line;
\end{lstlisting}
Note that the \lstinline!Line! primitive is also used to specify the graphical
representation of a connection.

For arrows:

\begin{itemize}
\item
  The arrow is drawn with an aspect ratio of 1/3 for each arrow part.
\item
  The arrowSize gives the width of the arrow (including the imagined
  other half for Half) so that lineThickness=10 and arrowSize=10 will
  touch at the outer parts.
\item
  All arrow variants overlap for overlapping lines.
\item
  The lines for the Open and Half variants are drawn with lineThickness.
\end{itemize}

\subsubsection{Polygon}\doublelabel{polygon}

A polygon is specified as follows:

\begin{lstlisting}[language=modelica]
record Polygon
  extends GraphicItem;
  extends FilledShape;
  Point points[:];
  Smooth smooth = Smooth.None "Spline outline";
end Polygon;
\end{lstlisting}
The polygon is automatically closed, if the first and the last points
are not identical.

\subsubsection{Rectangle}\doublelabel{rectangle}

A rectangle is specified as follows:

\begin{lstlisting}[language=modelica]
record Rectangle
  extends GraphicItem;
  extends FilledShape;
  BorderPattern borderPattern = BorderPattern.None;
  Extent extent;
  DrawingUnit radius = 0 "Corner radius";
end Rectangle;
\end{lstlisting}
The \lstinline!extent! attribute specifies the bounding box of the rectangle. If the
\lstinline!radius! attribute is specified, the rectangle is drawn with rounded
corners of the given radius.

\subsubsection{Ellipse}\doublelabel{ellipse}

An ellipse is specified as follows:

\begin{lstlisting}[language=modelica]
record Ellipse
  extends GraphicItem;
  extends FilledShape;
  Extent extent;
  Real startAngle(quantity="angle", unit="deg")=0;
  Real endAngle(quantity="angle", unit="deg")=360;
  EllipseClosure closure = if startAngle == 0 and endAngle == 360
  then EllipseClosure.Chord
  else EllipseClosure.Radial;
end Ellipse;
\end{lstlisting}
The \lstinline!extent! attribute specifies the bounding box of the ellipse.

Partial ellipses can be drawn using the \lstinline!startAngle! and \lstinline!endAngle!
attributes. These specify the endpoints of the arc prior to the stretch
and rotate operations. The arc is drawn counter-clockwise from
\lstinline!startAngle! to \lstinline!endAngle!, where startAngle and endAngle are defined
counter-clockwise from 3 o'clock (the positive x-axis).

The closure attribute specifies whether the endpoints specified by
\lstinline!startAngle! and \lstinline!endAngle! are to be joined by lines to the centre of the
extent (\lstinline!closure=EllipseClosure.Radial!), joined by a single straight line
between the end points (\lstinline!closure=EllipseClosure.Chord!), or left
unconnected (\lstinline!closure=EllipseClosure.None!). In the latter case, the
ellipse is treated as an open curve instead of a closed shape, and the
\lstinline!fillPattern! and \lstinline!fillColor! are not applied (if present, they are
ignored).

The default closure is \lstinline!EllipseClosure.Chord! when \lstinline!startAngle! is 0 and
\lstinline!endAngle! is 360, or \lstinline!EllipseClosure.Radial! otherwise.
\emph{{[}The
default for a closed ellipse is not} \lstinline!EllipseClosure.None!\emph{, since that
would result in} \lstinline!fillColor! and \lstinline!fillPattern! \emph{being ignored, making it
impossible to draw a filled ellipse.} \lstinline!EllipseClosure.Chord! \emph{is equivalent
in this case, since the chord will be of zero length.{]}}

\subsubsection{Text}\doublelabel{text}

A text string is specified as follows:

\begin{lstlisting}[language=modelica]
record Text
  extends GraphicItem;
  extends FilledShape;
  Extent extent;
  String textString;
  Real fontSize = 0 "unit pt";
  String fontName;
  TextStyle textStyle[:];
  Color textColor=lineColor;
  TextAlignment horizontalAlignment = TextAlignment.Center;
end Text;
\end{lstlisting}
The \lstinline!textColor! attribute defines the color of the text. The text is drawn
with transparent background and no border around the text (and without
outline). The contents inherited from \lstinline!FilledShape! is deprecated, but kept for compatibility reasons.

There are a number of common macros that can be used in the text, and
they should be replaced when displaying the text as follows:

\begin{itemize}
\item
  \%\emph{par} and \%\{\emph{par\}} replaced by the value of the parameter \lstinline!par!.
  If the value is numeric, tools shall display the value with \lstinline!displayUnit!, formatted according to bipm-specification.
  E.g., for
\begin{lstlisting}[language=modelica]
parameter Real t(unit="s", displayUnit="ms") = 0.1
\end{lstlisting}
  tools shall display \emph{100 ms}.
  The intent is that the text is easily readable,
  thus if \lstinline!par! is of an enumeration type, replace \lstinline!%par! by the item name,
  not by the full name.\\
  \emph{{[}Example: if \lstinline!par = "Modelica.Blocks.Types.Enumeration.Periodic"!,
  then \lstinline!%par! should be displayed as \emph{Periodic}{]}}\\
  The form \%\{\emph{par\}} allows component-references, and can be directly
  followed by a letter. Thus \lstinline!%{w}x%{h}! gives the value of \lstinline!w!
  directly followed by \emph{x} and the value of \lstinline!h! -- and \lstinline!%wxh! gives the value of the
  parameter \lstinline!wxh!. If the parameter does not exist it is an error.
\item
  \%\% replaced by \%
\item
  \%name replaced by the name of the component (i.e. the identifier for
  it in in the enclosing class).
\item
  \%class replaced by the name of the class (only the last part of the hierarchical name).
\end{itemize}

The style attribute \lstinline!fontSize! specifies the font size. If the \lstinline!fontSize!
attribute is 0 the text is scaled to fit its extent. Otherwise, the size
specifies the absolute size. The text is vertically centered in the
extent.

If the extent specifies a box with zero width and positive height the
height is used as height for the text (unless \lstinline!fontSize! attribute is
non-zero -- which specifies the absolute size), and the text is not
truncated (the horizontalAlignment is still used in this case).
\emph{{[}This is convenient for handling texts where the width is
unknown.{]}}

If the string \lstinline!fontName! is empty, the tool may choose a font. The font
names \lstinline!"serif"!, \lstinline!"sans-serif"!, and \lstinline!"monospace"! shall be recognized. If
possible the correct font should be used - otherwise a reasonable match,
or treat as if font-name was empty.

The style attribute \lstinline!textStyle! specifies variations of the font.

\subsubsection{Bitmap}\doublelabel{bitmap}

A bitmap image is specified as follows:

\begin{lstlisting}[language=modelica]
record Bitmap
  extends GraphicItem;
  Extent extent;
  String fileName "Name of bitmap file";
  String imageSource "Base64 representation of bitmap";
end Bitmap;
\end{lstlisting}
The \lstinline!Bitmap! primitive renders a graphical bitmap image. The data of the
image can either be stored on an external file or in the annotation
itself. The image is scaled to fit the extent. Given an extent \{\{x1,
y1\}, \{x2, y2\}\}, x2\textless{}x1 defines horizontal flipping and
y2\textless{}y1 defines vertical flipping around the center of the
object.

The graphical operations are applied in the order: scaling, flipping and
rotation.

When the attribute \lstinline!fileName! is specified, the string refers to an
external file containing image data. The mapping from the string to the
file is specified for some URIs in \autoref{external-resources}. The supported file
formats include \lstinline!PNG!, \lstinline!BMP! and \lstinline!JPEG!, other supported file formats are
unspecified.

When the attribute \lstinline!imageSource! is specified, the string contains the
image data -- and the image format is determined based on the contents.
The image is represented as a Base64 encoding of the image file format
(see RFC~4648, \url{http://tools.ietf.org/html/rfc4648}).

The image is uniformly scaled {[}\emph{to preserve aspect ratio}{]} so
it exactly fits within the extent {[}\emph{touching the extent along one
axis}{]}. The center of the image is positioned at the center of the
extent.

\subsection{Variable Graphics and Schematic Animation}\doublelabel{variable-graphics-and-schematic-animation}

Any value (coordinates, color, text, etc.) in graphical annotations can
be dependent on class variables using the \lstinline!DynamicSelect! expression.
\lstinline!DynamicSelect! has the syntax of a function call with two arguments,
where the first argument specifies the value of the editing state and
the second argument the value of the non-editing state. The first
argument must be a literal expression. The second argument may contain
references to variables to enable a dynamic behavior.

{[}\emph{Example}: \emph{The level of a tank could be animated by a
rectangle expanding in vertical direction and its color depending on a
variable overflow:}
\begin{lstlisting}[language=modelica]
  annotation (
   Icon(graphics={Rectangle(
     extent=DynamicSelect({{0,0},{20,20}},{{0,0},{20,level}}),
      fillColor=DynamicSelect({0,0,255},
                              if overflow then {255,0,0} else {0,0,255}))}
 );
\end{lstlisting}
{]}

\subsection{User input}\doublelabel{user-input}

It is possible to interactively modify variables during a simulation.
The variables may either be parameters, discrete variables or states.
New numeric values can be given, a mouse click can change a Boolean
variable or a mouse movement can change a Real variable. Input fields
may be associated with a \lstinline!GraphicItem! or a component as an array named
\lstinline!interaction!. The \lstinline!interaction! array may occur as an attribute of a
graphic primitive, an attribute of a component annotation or as an
attribute of the layer annotation of a class.

\subsubsection{Mouse input}\doublelabel{mouse-input}

A Boolean variable can be changed when the cursor is held over a
graphical item or component and the selection button is pressed if the
interaction annotation contains \lstinline!OnMouseDownSetBoolean!:

\begin{lstlisting}[language=modelica]
record OnMouseDownSetBoolean
  Boolean variable "Name of variable to change when mouse button pressed";
  Boolean value "Assigned value";
end OnMouseDownSetBoolean;
\end{lstlisting}
{[}\emph{Example}: \emph{A button can be represented by a rectangle
changing color depending on a Boolean variable} on \emph{and toggles the
variable when the rectangle is clicked on:}

\begin{lstlisting}[language=modelica]
  annotation (Icon(graphics={Rectangle(extent=[0,0; 20,20],
  fillColor=if  on then {255,0,0} else
  {0,0,255})},
  interaction={ OnMouseDownSetBoolean (on, not on)}));
\end{lstlisting}
{]}

In a similar way, a variable can be changed when the mouse button is
\emph{released}:

\begin{lstlisting}[language=modelica]
record OnMouseUpSetBoolean
  Boolean variable "Name of variable to change when mouse button released";
  Boolean value "Assigned value";
end OnMouseUpSetBoolean;
\end{lstlisting}
Note that several interaction objects can be associated with the same
graphical item or component.

{[}\emph{Example}:
\begin{lstlisting}[language=modelica]
interaction={ OnMouseDownSetBoolean(on, true), OnMouseUpSetBoolean(on, false)};
\end{lstlisting}
{]}

The \lstinline!OnMouseMoveXSetReal! interaction object sets the variable to the
position of the cursor in X direction in the local coordinate system
mapped to the interval defined by the \lstinline!minValue! and \lstinline!maxValue! attributes.

\begin{lstlisting}[language=modelica]
record OnMouseMoveXSetReal
  Real xVariable "Name of variable to change when cursor moved in x direction";
  Real minValue;
  Real maxValue;
end OnMouseMoveXSetReal;
\end{lstlisting}
The \lstinline!OnMouseMoveYSetReal! interaction object works in a corresponding way
as the \lstinline!OnMouseMoveXSetReal! object but in the Y direction.

\begin{lstlisting}[language=modelica]
record OnMouseMoveYSetReal
  Real yVariable "Name of variable to change when cursor moved in y direction";
  Real minValue;
  Real maxValue;
end OnMouseMoveYSetReal;
\end{lstlisting}
\subsubsection{Edit input}\doublelabel{edit-input}

The \lstinline!OnMouseDownEditInteger! interaction object presents an input field
when the graphical item or component is clicked on. The field shows the
actual value of the variable and allows changing the value. If a too
small or too large value according to the \lstinline!min! and \lstinline!max! parameter values
of the variable is given, the input is rejected.

\begin{lstlisting}[language=modelica]
record OnMouseDownEditInteger
  Integer variable "Name of variable to change";
end OnMouseDownEditInteger;
\end{lstlisting}
The \lstinline!OnMouseDownEditReal! interaction object presents an input field when
the graphical item or component is clicked on. The field shows the
actual value of the variable and allows changing the value. If a too
small or too large value according to the \lstinline!min! and \lstinline!max! parameter values
of the variable is given, the input is rejected.

\begin{lstlisting}[language=modelica]
record OnMouseDownEditReal
  Real variable "Name of variable to change";
end OnMouseDownEditReal;
\end{lstlisting}
The \lstinline!OnMouseDownEditString! interaction object presents an input field
when the graphical item or component is clicked on. The field shows the
actual value of the variable and allows changing the value.

\begin{lstlisting}[language=modelica]
record OnMouseDownEditString
  String variable "Name of variable to change";
end OnMouseDownEditString;
\end{lstlisting}
\section{Annotations for the Graphical User Interface}\doublelabel{annotations-for-the-graphical-user-interface}

A class may have the following annotations to define properties of the
graphical user interface:
\begin{lstlisting}[language=modelica]
 annotation(defaultComponentName = "name")
\end{lstlisting}

When creating a component of the given class, the recommended component
name is name.
\begin{lstlisting}[language=modelica]
  annotation(defaultComponentPrefixes = "prefixes")
\end{lstlisting}

When creating a component, it is recommended to generate a declaration
of the form
\begin{lstlisting}[language=modelica]
  prefixes class-name component-name
\end{lstlisting}

The following prefixes may be included in the string prefixes: inner,
outer, replaceable, constant, parameter, discrete. {[}\emph{In
combination with} defaultComponentName \emph{it can be used to make it
easy for users to create} inner \emph{components matching the} outer
\emph{declarations; see also example below. If the prefixes contain} inner \emph{or} outer
\emph{and the default name cannot be used (e.g., since it is already in use) it is recommended to give a diagnostic.}{]}
\begin{lstlisting}[language=modelica]
  annotation(missingInnerMessage = "message")
\end{lstlisting}

When an \lstinline!outer! component of the class does not have a corresponding \lstinline!inner!
component, the literal string message may be used as part of a diagnostic message (together with appropriate context), see
\autoref{instance-hierarchy-name-lookup-of-inner-declarations}.

{[}\emph{Example}:

\begin{lstlisting}[language=modelica]
model World
  ...
  annotation(defaultComponentName = "world",
  defaultComponentPrefixes = "inner replaceable",
  missingInnerMessage = "The World object is missing");
end World;
\end{lstlisting}
When an instance of model \lstinline!World! is dragged in to the diagram layer, the
following declaration is generated:
\begin{lstlisting}[language=modelica]
  inner replaceable World world;
\end{lstlisting}

{]}

A simple type or component of a simple type may have:
\begin{lstlisting}[language=modelica]
  annotation(absoluteValue=false);
\end{lstlisting}

If \lstinline!false!, then the variable defines a relative quantity, and if true an
absolute quantity. {[}\emph{When converting between units (in the
user-interface for plotting and entering parameters), the offset must be
ignored, for a variable defined with annotation absoluteValue = false.
This annotation is used in the Modelica Standard Library for example in
Modelica.SIunits for the type definition TemperatureDifference.}{]}

A model or block definition may contain:
\begin{lstlisting}[language=modelica]
  annotation(defaultConnectionStructurallyInconsistent=true)
\end{lstlisting}

If \lstinline!true!, it is stated that a default connection will result in a
structurally inconsistent model or block\footnote{For the precise
  definition of "structurally inconsistent" see the article:
  \href{http://epubs.siam.org/doi/abs/10.1137/0909014}{Pantelides C.C.:
  The Consistent Initialization of Differential-Algebraic Systems, SIAM
  J. Sci. and Stat. Comput. Volume 9, Issue 2, pp. 213--231 (March
  1988)}}. A "default connection" is constructed by instantiating the
respective \lstinline!model! or \lstinline!block! and for every input \lstinline!u! providing an equation
\lstinline!0=f(u)!, and for every (potential,flow) pair of the form \lstinline!(v,i)!, providing
an equation of the form \lstinline!0=f(v,i)!.

{[}\emph{It is useful to check all models/blocks of a Modelica package
in a simple way. One check is to default connect every model/block and
to check whether the resulting class is structurally consistent (which is a
stronger requirement than being locally balanced). It is rarely needed; but is for
example used in Modelica.Blocks.Math.InverseBlockConstraints, in order
to prevent a wrong error message. Additionally, when a user defined
model is structurally inconsistent, a tool should try to pinpoint in
which class the error is present. This annotation avoids then to show a
wrong error message.}{]}

A class may have the following annotation:
\begin{lstlisting}[language=modelica]
  annotation(obsolete = "message");
\end{lstlisting}

It indicates that the class ideally should not be used anymore and gives
a message indicating the recommended action.
This annotation is not inherited, the assumption is that if a class uses
an obsolete class (as a base-class or as the class of one of the components)
that shall be updated - ideally without impacting users of the class.
If that is not possible the current class can have also have an obsolete-annotation.

A declaration may have the following annotations:
\begin{lstlisting}[language=modelica]
  annotation(unassignedMessage = "message");
\end{lstlisting}

When the variable to which this annotation is attached in the
declaration cannot be computed due to the structure of the equations,
the string message can be used as a diagnostic message. {[}\emph{When
using BLT partitioning, this means if a variable \lstinline!a! or one of its
aliases \lstinline!b = a! or \lstinline!b = -a! cannot be assigned, the message is
displayed. This annotation is used to provide library specific error
messages.}{]}

{[}\emph{Example}:

\begin{lstlisting}[language=modelica]
connector Frame "Frame of a mechanical system"
  ...
  flow Modelica.SIunits.Force f[3]
  annotation(unassignedMessage =
      "All Forces cannot be uniquely calculated. The reason could be that the
     mechanism contains a planar loop or that joints constrain the same motion.
     For planar loops, use in one revolute joint per loop the option
     PlanarCutJoint=true in the Advanced menu.
     ");
end Frame;
\end{lstlisting}
{]}
\begin{lstlisting}[language=modelica]
annotation(Dialog(enable = true, tab = "General",
                  group = "Parameters",
                  showStartAttribute = false,
                  colorSelector = false,
                  groupImage="modelica://MyPackage/Resources/Images/switch.png",
                  connectorSizing = false));
\end{lstlisting}

The annotations \lstinline!tab! and \lstinline!group! define the placement of
the component or of variables in a dialog with optional tab and group
specification. If \lstinline!enable = false!, the input field may
be disabled {[}\emph{and no input can be given}{]}. If
\lstinline!showStartAttribute = true! the dialog should allow the user to
set the start-value and the fixed attribute for the variable instead of
the value-attribute {[}\emph{this is primarily intended for
non-parameter values and avoids introducing a separate parameter for the
start-value of the variable{]}}.

If \lstinline!colorSelector=true!, it indicates that an rgb-value selector can be
presented for a vector of three elements and generate values 0..255 (the
annotation should be useable both for vectors of Integers and Reals).

The annotation \lstinline!groupImage! references an image using an URI (see
\autoref{external-resources}), and the image is intended to be shown together with the
parameter-group (only one image per group is supported). Disabling the
input field will not disable the image.

The background of the \lstinline!groupImage! and any image used in HTML-documentation is recommended to be transparent (intended to be a light color) - or white.

The value of the connectorSizing annotation must be a literal
\lstinline!false! or \lstinline!true! value {[}\emph{since if the value is an
expression, the connectorSizing functionality is conditional and this
will then lead easily to wrong models}{]}. If \lstinline!connectorSizing = false!, this annotation has no effect.
If \lstinline!connectorSizing = true!, the corresponding variable must be declared with the
\lstinline!parameter! prefix, must be a subtype of a scalar Integer and
must have a literal default value of zero {[}\emph{since this annotation
is designed for a parameter that is used as vector dimension and the
dimension of the vector should be zero when the component is dragged or
redeclared; furthermore, when a tool does not support the
connectorSizing annotation, dragging will still result in a correct
model}{]}.

If \lstinline!connectorSizing = true!, a tool may set the parameter value
in a modifier automatically, if used as dimension size of a vector of
connectors. {[}\emph{The connectorSizing annotation is used in cases
where connections to a vector of connectors shall be made and a new
connection requires to resize the vector and to connect to the new index
(unary connections). The annotation allows a tool to perform these two
actions in many cases automatically. This is, e.g., very useful for
state machines and for certain components of fluid libraries.}{]}

Annotation \lstinline!Dialog! is defined as:

\begin{lstlisting}[language=modelica]
record Dialog
  parameter String tab = "General";
  parameter String group = "Parameters";
  parameter Boolean enable = true;
  parameter Boolean showStartAttribute = false;
  parameter Boolean colorSelector = false;
  parameter Selector loadSelector;
  parameter Selector saveSelector;
  parameter String groupImage = "";
  parameter Boolean connectorSizing = false;
end Dialog;

record Selector
  parameter String filter="";
  parameter String caption="";
end Selector;
\end{lstlisting}
A parameter dialog is a sequence of tabs with a sequence of groups
inside them.

A \lstinline!Selector! displays a file dialog to select a file. Setting \lstinline!filter! only
shows files that fulfill the given pattern defined by \lstinline!text1 (*.ext1);;text2 (*.ext2)! to show only files with file extension
\filename{ext1} or \filename{ext2} and displaying a description text \lstinline!text1! and
\lstinline!text2!, respectively. Parameter \lstinline!caption! is the text displayed in the
dialog menu. Parameter \lstinline!loadSelector! is used to select an existing file
for reading, whereas parameter \lstinline!saveSelector! is used to define a file for
writing.

{[}\emph{Example}:

\begin{lstlisting}[language=modelica]
model DialogDemo
  parameter Boolean b = true "Boolean parameter";
  parameter Modelica.SIunits.Length length "Real parameter with unit";
  parameter Integer nInports=0
     annotation(Dialog(connectorSizing=true));
  parameter Real r1 "Real parameter in Group 1"
     annotation(Dialog(group="Group 1"));
  parameter Real r2 "Disabled Real parameter in Group 1"
     annotation(Dialog(group="Group 1",enable = not b));
  parameter Real r3 "Real parameter in Tab 1"
     annotation(Dialog(tab="Tab 1"));
  parameter Real r4 "Real parameter in Tab 1 and Group 2"
     annotation(Dialog(tab="Tab 1", group="Group 2"));
  StepIn stepIn[nInports];
  ...
end DialogDemo;
\end{lstlisting}
\emph{When clicking on an instance of model DialogDemo, a menu pops up
that may have the following layout (other layouts are also possible,
this is vendor specific). Note, parameter nInports is not present in the
menu since it has the \lstinline!connectorSizing! annotation and therefore it
should not be modified by the user (an alternative is to show parameter
nInports in the menu but with disabled input field): }

\includegraphics[width=2.25in,height=1.125in]{disabledparameter}
\includegraphics[width=2.625in,height=0.89583in]{tabparameter}

\emph{The following part is non-normative text and describes a useful
way to handle the connectorSizing annotation in a tool (still a tool may
use another strategy and/or may handle other cases than described
below). The recommended rules are clarified at hand of the following
example which represents a connector and a model from the
Modelica.StateGraph library:}

\begin{lstlisting}[language=modelica]
connector StepIn // Only 1:1 connections are possible since input used
  output Boolean occupied;
  input Boolean set;
end StepIn;

model Step
  // nIn cannot be set in the parameter dialog (but maybe shown)
  parameter Integer nIn=0
  annotation(Dialog(connectorSizing=true));
  StepIn inPorts[nIn];
  ...
end Step;
\end{lstlisting}
\emph{If the parameter is used as dimension size of a {vector of
connectors}, it is automatically updated according to the following
rules:}

\begin{enumerate}
\item
  \emph{If a new connection line is drawn between one outside and one
  inside vector of connectors both dimensioned with (connectorSizing)
  parameters, a connection between the two vectors is performed and the
  (connectorSizing) parameter is propagated from connector to component.
  Other types of outside connections do not lead to an automatic update
  of a (connectorSizing) parameter. Example:} \emph{Assume there is a
  connector inPorts and a component step1:}
\begin{lstlisting}[language=modelica]
  parameter Integer nIn=0 annotation(Dialog(connectorSizing=true));
  StepIn inPorts[nIn];
  Step step1(nIn=0);
\end{lstlisting}
  \emph{Drawing a connection line between connectors inPorts and
  step1.inPorts results in:}
\begin{lstlisting}[language=modelica]
  parameter Integer nIn=0 annotation(Dialog(connectorSizing=true));
  StepIn inPorts[nIn];
  Step step1(nIn=nIn); // nIn=0 changed to nIn=nIn
  equation
  connect(inPorts, step1.inPorts); // new connect equation
\end{lstlisting}
\item
  \emph{If a connection line is deleted between one outside and one
  inside vector of connectors both dimensioned with (connectorSizing)
  parameters, the connect equation is removed and the (connectorSizing)
  parameter of the component is set to zero or the modifier is removed}.
  \emph{Example:} \emph{Assume the connection line in (3) is removed.
  This results in:}
\begin{lstlisting}[language=modelica]
  parameter Integer nIn=0 annotation(Dialog(connectorSizing=true));
  StepIn inPorts[nIn];
  Step step1; // modifier nIn=nIn is removed
\end{lstlisting}
\item
  \emph{If a new connection line is drawn to an inside connector with
  connectorSizing and case 1 does not apply then, the parameter is
  incremented by one and the connection is performed for the new highest
  index. Example:} \emph{Assume that 3 connections are present and a new
  connection is performed. The result is:}
\begin{lstlisting}[language=modelica]
    Step step1(nIn=4); // index changed from nIn=3 to nIn=4
  equation
    connect(.., step1.inPorts[4]); // new connect equation
\end{lstlisting}
  \emph{In some applications, like state machines, the vector index is
  used as a priority, e.g., to define which transition is firing if
  several transitions become active at the same time instant. It is then
  not sufficient to only provide a mechanism to always connect to the
  last index. Instead, some mechanism to select an index conveniently
  should be provided. }
\item
  \emph{If a connection line is deleted to an inside connector with
  connectorSizing and case 2 does not apply then, then the
  (connectorSizing) parameter is decremented by one and all connections
  with index above the deleted connection index are also decremented by
  one}. \emph{Example:} \emph{Assume there are 4 connections:}
\begin{lstlisting}[language=modelica]
  Step step1(nIn=4);
equation
  connect(a1, step1.inPorts[1]);
  connect(a2, step1.inPorts[2]);
  connect(a3, step1.inPorts[3]);
  connect(a4, step1.inPorts[4]);
\end{lstlisting}
  \emph{and the connection from a2 to step1. inPorts{[}2{]} is deleted.
  This results in}
\begin{lstlisting}[language=modelica]
  Step step1(nIn=3);
equation
  connect(a1, step1.inPorts[1]);
  connect(a3, step1.inPorts[2]);
  connect(a4, step1.inPorts[3]);
\end{lstlisting}
\end{enumerate}

\emph{These rules also apply if the connectors and/or components are
defined in {superclass}. Example:} \emph{Assume that step1 is defined in
superclass CompositeStep with 3 connections, and a new connection is
performed in a subclass. The result is:}
\begin{lstlisting}[language=modelica]
  extends CompositeStep(step1(nIn=4)); // new modifier nIn=4
equation
  connect(.., step1.inPorts[4]);  // new connect equation
\end{lstlisting}
{]}

\section{Annotations for Version Handling}\doublelabel{annotations-for-version-handling}

A top-level package or model can specify the version of top-level
classes it uses, its own version number, and if possible how to convert
from previous versions. This can be used by a tool to guarantee that
consistent versions are used, and if possible to upgrade usage from an
earlier version to a current one.

\subsection{Version Numbering}\doublelabel{version-numbering}

Version numbers are of the forms:

\begin{itemize}
\item
  Main release versions: """ \lstinline!UNSIGNED-INTEGER! \{ "." \lstinline!UNSIGNED-INTEGER! \}
  """ {[}\emph{Example: "2.1"}{]}
\item
  Pre-release versions: """ \lstinline!UNSIGNED-INTEGER! \{ "." \lstinline!UNSIGNED-INTEGER! \}
  " " \{\lstinline!S-CHAR!\} """ {[}\emph{Example: "2.1 Beta 1"}{]}
\item
  Un-ordered versions: """ \lstinline!NON-DIGIT! \{\lstinline!S-CHAR!\} """\\
  {[}\emph{Example: "Test 1"}{]}
\end{itemize}

The main release versions are ordered using the hierarchical numerical
names, and follow the corresponding pre-release versions. The
pre-release versions of the same main release version are internally
ordered alphabetically.

\subsection{Version Handling}\doublelabel{version-handling}

In a top-level class, the version number and the dependency to earlier
versions of this class are defined using one or more of the following
annotations:

\begin{itemize}
\item
  \lstinline!version = CURRENT-VERSION-NUMBER!\\
  Defines the version number of the model or package. All classes within
  this top-level class have this version number.
\item
  \lstinline!conversion(noneFromVersion = VERSION-NUMBER)!\\
  Defines that user models using the \lstinline!VERSION-NUMBER! can be upgraded to
  the \lstinline!CURRENT-VERSION-NUMBER! of the current class without any changes.
\item
  \lstinline!conversion(from(version = Versions, [to=VERSION-NUMBER,] Convert))!\\
  where \emph{Versions} is \lstinline!VERSION-NUMBER! \textbar{}
   \lstinline!{VERSION-NUMBER,VERSION-NUMBER, ...}!\
  and \lstinline!Convert! is \lstinline!script="..."! \textbar{}
   \lstinline!change={conversionRule(), ..., conversionRule()}!\\*[.5ex]
  Defines that user models using the \lstinline!VERSION-NUMBER! or any of the given
  \lstinline!VERSION-NUMBER! can be upgraded to the given \lstinline!VERSION-NUMBER! (if the
  to-tag is missing this is the \lstinline!CURRENT-VERSION-NUMBER!) of the current
  class by applying the given conversion rules. The script consists of
  an unordered sequence of  \lstinline!conversionRule();! (and optionally Modelica
  comments). The  \lstinline!conversionRule! functions are defined in \autoref{conversion-rules}.
  \emph{{[}The to-tag is added for clarity and optionally allows a tool
  to convert in multiple steps.{]}}
\item
  \lstinline!uses(IDENT (version = VERSION-NUMBER [, versionBuild=INTEGER] [, dateModified=STRING] ) )!\\
  Defines that classes within this top-level class uses version
  \lstinline!VERSION-NUMBER! of classes within the top-level class \lstinline!IDENT!.
\end{itemize}

The annotations \lstinline!uses! and \lstinline!conversion! may contain several different
sub-entries.

{[}\emph{Example}:

\begin{lstlisting}[language=modelica]
package Modelica
  ...
  annotation(version="3.1",
  conversion(noneFromVersion="3.1 Beta 1",
  noneFromVersion="3.1 Beta 2",
  from(version={"2.1", "2.2", "2.2.1"},
  script="convertTo3.mos"),
  from(version="1.5",
  script="convertFromModelica1_5.mos")));
end Modelica;

model A
  ...
  annotation(version="1.0",
  uses(Modelica(version="1.5")));
end A;

model B
  ...
  annotation(uses(Modelica(version="3.1 Beta 1")));
end B;
\end{lstlisting}
\emph{In this example the model} \lstinline!A! \emph{uses an older version of the
Modelica library and can be upgraded using the given script, and model}
\lstinline!B! \emph{uses an older version of the Modelica library but no changes are
required when upgrading. }

{]}

\subsubsection{Conversion rules}\doublelabel{conversion-rules}

There are a number of functions: convertClass, convertClassIf,
convertElement, convertModifiers, convertMessage defined as follows. The
calls of these functions do not directly convert, instead they define
conversion rules as below. The order between the function calls does not
matter, instead the longer paths (in terms number of hierarchical names)
are used first as indicated below, and it is an error if there are any
ambiguities.

These functions can be called with literal strings or array of strings
and vectorize according to \autoref{scalar-functions-applied-to-array-arguments}.

All of these convert-functions only use inheritance among user
models, and not in the library that is used for the conversion -- thus
conversions of base-classes will require multiple conversion-calls; this
ensures that the conversion is independent of the new library structure.
The class-name used as argument to convertElement and convertModifiers
is similarly the old name of the class, i.e. the name before it is
possibly converted by convertClass. {[}\emph{This allows the conversion
to be done without access to the old version of the library (by suitable
modifications of the lookup). Another alternative is to use the old
version of the library during the conversion.}{]}

\paragraph*{convertClass("OldClass", "NewClass")}\doublelabel{convertclassoldclassnewclass}

Convert class \lstinline!OldClass! to \lstinline!NewClass!.

Match longer path first, so if converting both \lstinline!A! to \lstinline!C! and \lstinline!A.B! to \lstinline!D! then
\lstinline!A.F! is converted to \lstinline!C.F! and \lstinline!A.B.E! to \lstinline!D.E!. This is considered before
convertMessage for the same \lstinline!OldClass!.

\paragraph*{convertClassIf("OldClass", "oldElement", "whenValue","NewClass")}\doublelabel{convertclassifoldclass-oldelement-whenvalue-newclass}

Convert class \lstinline!OldClass! to \lstinline!NewClass! if the literal modifier for
\lstinline!oldElement! has the value \lstinline!whenValue!, and also remove the modifier for
\lstinline!oldElement!.

These are considered before \lstinline!convertClass! and \lstinline!convertMessage! for the same
\lstinline!OldClass!.

\paragraph*{convertElement("OldClass", "OldName", "NewName")}\doublelabel{convertelementoldclassoldnamenewname}

In \lstinline!OldClass! convert element \lstinline!OldName! to \lstinline!NewName!.
Both \lstinline!OldName! and \lstinline!NewName!
normally refer to components -- but they may also refer to
class-parameters, or hierarchical names. For hierarchical names the
longest match is used first.

\paragraph*{convertModifiers}\doublelabel{convertmodifiers}

\begin{lstlisting}[language=modelica]
convertModifiers("OldClass",
{"OldModifier1=default1", "OldModifier2=default2", ...},
{"NewModifier1=...%OldModifier1\%"} [, simplify=true] )
\end{lstlisting}

Normal case; if any modifier among \lstinline!OldModifier! exist then replace all of
them with the \lstinline!NewModifiers!. The defaults (if present) are used if there
are multiple \lstinline!OldModifier! and not all are set in the component instance.

If \lstinline!simplify! is specified and true then perform obvious simplifications
to clean up the new modifier; otherwise leave as is.
\begin{nonnormative}
Note: \lstinline!simplify! is primarily intended for converting enumerations and emulated
enumerations that naturally lead to large nested if-expressions. The
simplifications may also simplify parts of the original expression.
\end{nonnormative}

If the modifiers contain literal string values they must be quoted.

Behaviour in unusual cases:

\begin{itemize}
\item
  if NewModifier list is empty then the modifier is just removed
\item
  If OldModifer list is empty it is added for all uses of the class
\item
  If OldModifier\_i is cardinality(a)=0 the conversion will only be
  applied for a component comp if there are no inside connections to
  comp.a. This can be combined with other modifiers that are handled in
  the usual way.
\item
  If OldModifier\_i is cardinality(a)=1 the conversion will only be
  applied for a component comp if there are any inside connections to
  comp.a.
\end{itemize}

The converted modifiers and existing modifiers are merged such that the existing modifiers take precedence over the result of convertModifiers.
A diagnostics is recommended if this merging removes some modifiers unless those modifiers are identical or it is the special case of an empty OldModifier list.
\begin{nonnormative}
This can be used to handle the case where the default value was changed.
\end{nonnormative}

Converting modifiers with cardinality is used to remove the deprecated
operator cardinality from model libraries, and replace tests on
cardinality in models by parameters explicitly enabling the different
cases. {[}\emph{I.e. instead of model A internally testing if its
connector B is connected, there will be a parameter for enabling
connector B, and the conversion ensures that each component of model A
will have this parameter set accordingly.}{]} The case where the old
class is used as a base-class, and there are any outside connections to
a, and there is convertModifiers involving the cardinality of a is not
handled. {[}\emph{In case a parameter is simply renamed it is preferable
to use convertElement, since that also handles e.g. binding equations
using the parameter.}{]}

\paragraph*{convertMessage("OldClass", "Failed Message");}\doublelabel{convertmessageoldclass-failed-message}

For any use of \lstinline!OldClass! (or element of \lstinline!OldClass!) report that conversion
could not be applied with the given message. \emph{{[}This may be useful
if there is no possibility to convert a specific class. An alternative
is to construct ObsoleteLibraryA for problematic cases, which may be
more work but allows users to directly run the models after the
conversion and later convert them.{]}}

\subsection{Mapping of Versions to File System}\doublelabel{mapping-of-versions-to-file-system}

A top-level class, \lstinline!IDENT!, with version \lstinline!VERSION-NUMBER! can be stored in
one of the following ways in a directory given in the \lstinline!MODELICAPATH!:

\begin{itemize}
\item
  The file \lstinline!IDENT ".mo"! {[}\emph{Example:} \filaname{Modelica.mo}{]}
\item
  The file \lstinline!IDENT " " VERSION-NUMBER ".mo"! {[}\emph{Example:} \filename{Modelica 2.1.mo}{]}
\item
  The directory \lstinline!IDENT! {[}\emph{Example:} \lstinline!Modelica!{]} with the file
  \filename{package.mo} directly inside it
\item
  The directory \lstinline!IDENT " " VERSION-NUMBER! {[}\emph{Example:}
	\filename{Modelica 2.1}{]} with the file \filename{package.mo} directly inside it
\end{itemize}

This allows a tool to access multiple versions of the same package.

\subsection{Version Date and Build Information}\doublelabel{version-date-and-build-information}

Besides version information, a top level class can have additionally the
following top-level annotations to specify associated information to the
version number:

\begin{lstlisting}[language=modelica]
String versionDate   "UTC date of first version build (in format: YYYY-MM-DD)";
Integer versionBuild "Larger number is a more recent maintenance update";
String dateModified  "UTC date and time of the latest change to the package in the
                      following format (with one space between date and time):
                      YYYY-MM-DD hh:mm:ssZ";
String revisionId    "Revision identifier of the version management system used
                      to manage this library. It marks the latest submitted change to
                      any file belonging to the package";
\end{lstlisting}
{[}\emph{Example:}

\begin{lstlisting}[language=modelica]
package Modelica
  ...
  annotation(version = "3.0.1",
  versionDate = "2008-04-10",
  versionBuild = 4,
  dateModified = "2009-02-15 16:33:14Z",
  revisionId = "$Id:: package.mo 2566 2009-05-26 13:25:54Z #$");
end Modelica;

model M1
  annotation(uses(Modelica(version = "3.0.1"))); // Common case
end M1

model M2
  annotation(uses(Modelica(version = "3.0.1", versionBuild = 4)));
end M2
\end{lstlisting}
{]}

The meaning of these annotation is:

\begin{itemize}
\item
  \lstinline!version! is the version number of the released library,
  see \autoref{version-handling}.
\item
  \lstinline!versionDate! is the date in UTC format (according to ISO
  8601) when the library was released. This string is updated by the
  library author to correspond with the version number.
\item
  \lstinline!versionBuild! is the optional build number of the library.
  When a new version is released \lstinline!versionBuild! should be omitted or
  \lstinline!versionBuild = 1!. There might be bug fixes to the library that do
  not justify a new library version. Such maintenance changes are called
  a \emph{build} release of the library. For every new maintenance change,
  the \lstinline!versionBuild! number is increased. A \lstinline!versionBuild! number $A$
  that is higher than \lstinline!versionBuild! number $B$, is a newer release of the
  library. There are no conversions between the same versions with
  different build numbers. \emph{\\}Two releases of a library with the same \lstinline!version! but different
  \lstinline!versionBuild! are in general assumed to be compatible. In special
  cases, the uses clause of a model may specify \lstinline!versionBuild! and/or
  \lstinline!dateModified! {[}\emph{in such a case the tool is expected to give
  a warning if there is a mismatch between library and model}{]}\emph{.}
\item
  \lstinline!dateModified! is the UTC date and time (according to ISO
  8601) of the last modification of the package. {[}\emph{The intention
  is that a Modelica tool updates this annotation whenever the package
  or part of it was modified and is saved on persistent storage (like
  file or database system).}{]}
\item
  \lstinline!revisionId! is a tool specific revision identifier
  possibly generated by a source code management system (e.g. Subversion
  or CVS). This information allows to exactly identify the library
  source code in the source code management system.
\end{itemize}

The versionBuild and dateModified annotations can also be specified in
the \lstinline!uses! annotation (together with the version number). {[}\emph{The
recommendation is that they are not stored in the uses annotation
automatically by a tool.}{]}

\section{Annotations for Access Control to Protect Intellectual Property}\doublelabel{annotations-for-access-control-to-protect-intellectual-property}

This section presents annotations to define the protection and the
licensing of packages. The goal is to unify basic mechanisms to control
the access to a package in order to protect the intellectual property
contained in it. This information is used to encrypt a package and bind
it optionally to a particular target machine, and/or restrict the usage
for a particular period of time.

{[}\emph{Protecting the intellectual property of a Modelica package is
considerably more difficult than protecting code from a programming
language. The reason is that a Modelica tool needs the model equations
in order that it can process the equations symbolically, as needed for
acausal modeling. Furthermore, if a Modelica tool generates C-code of
the processed equations, this code is then potentially available for
inspection by the user. Finally, the Modelica tool vendors have to be
trusted, that they do not have a backdoor in their tools to store the
(internally) decrypted classes in human readable format. The only way to
protect against such misuse is legally binding warranties of the tool
vendors. }

\emph{The intent of this section is to enable a library vendor to
maintain one source version of their Modelica library that can be
encrypted and used with several different Modelica tools, using
different encryption formats.}{]}

\textbf{Definitions:}

\begin{longtable}[]{|p{2.5cm}|p{12cm}|}
\hline
\emph{Term} & \emph{Description}\\ \hline
\endhead
\textbf{Protection} & Define what parts of a class are
visible.\\ \hline
\textbf{Obfuscation} & Changing a Modelica class or generated code so
that it is difficult to inspect by a user {[}\emph{(e.g. by
automatically renaming variables to non-meaningful
names).}{]}\\ \hline
\textbf{Encryption} & Encoding of a model or a package in a form so that
the modeler cannot inspect any content of a class without an appropriate
key. An encrypted package that has the \lstinline!Protection! annotation is
read-only; the way to modify it is to generate a new encrypted
version.\\ \hline
\textbf{Licensing} & Restrict the use of an encrypted package for
particular users for a specified period of time.\\ \hline

\end{longtable}

In this section annotations are defined for protection and
licensing. Obfuscation and encryption are not standardized.
protection and licensing are both defined inside the
\lstinline!Protection! annotation:

\begin{lstlisting}[language=modelica]
annotation(Protection(...));
\end{lstlisting}

\subsection{Protection of Classes}\doublelabel{protection-of-classes}

A class may have the following annotations to define what parts of a
class are visible, and only the parts explicitly listed as visible below can be accessed
(if a class is encrypted and no Protection annotation is defined, the access annotation has the default value
\lstinline!Access.documentation!):

\begin{lstlisting}[language=modelica]
  type Access = enumeration(hide, icon, documentation,
  diagram, nonPackageText, nonPackageDuplicate,
  packageText, packageDuplicate);
  annotation(Protection(access = Access.documentation));
\end{lstlisting}
The items of the Access enumeration have the following meaning:

\begin{enumerate}
\item
  \lstinline!Access.hide!\\
  Do not show the class anywhere (it is not possible to inspect any part
  of the class).
\item
  \lstinline!Access.icon!\textbf{\\
  }The class can be instantiated and public parameter, constant, input,
  output variables as well as public connectors can be accessed, as well
  as the icon annotation, as defined in \autoref{annotations-for-graphical-objects} (the declared
  information of these elements can be shown). Additionally, the class
  name and its description text can be accessed.
\item
  \lstinline!Access.documentation!\\
  Same as \lstinline!Access.icon! and additionally the documentation annotation (as
  defined in \autoref{annotations-for-documentation}) can be accessed. HTML-generation in the
  documentation annotation is normally performed before encryption, but
  the generated HTML is intended to be used with the encrypted package.
  Thus the HTML-generation should use the same access as the encrypted
  version-- even before encryption.
\item
  \lstinline!Access.diagram!\\
  Same as \lstinline!Access.documentation! and additionally, the diagram annotation,
  and all components and connect equations that have a graphical
  annotation can be accessed.
\item
  \lstinline!Access.nonPackageText!\\
  Same as \lstinline!Access.diagram! and additionally if it is not a package: the
  whole class definition can be accessed (but that text cannot be copied, i.e., you can see but not copy the source code).
\item
  \lstinline!Access.nonPackageDuplicate!\\
  Same as \lstinline!Access.nonPackageText! and additionally if it is not a package:
  the class, or part of the class, can be copied.
\item
  \lstinline!Access.packageText!\\
  Same as \lstinline!Access.diagram! (note: \textbf{not} including all rights of
  \lstinline!Access.nonPackageDuplicate!) and additionally the whole class
  definition can be accessed (but that text cannot be copied, i.e., you can see but not copy the source code).
\item
  \lstinline!Access.packageDuplicate!\\
  Same as \lstinline!Access.packageText! and additionally the class, or part of the
  class, can be copied.
\end{enumerate}

The\lstinline!access! annotation holds for the respective class and all classes
that are hierarchically on a lower level, unless overriden by a
Protection annotation with \lstinline!access! {[}\emph{e.g. if the annotation is
given on the top level of a package and at no other class in this
package, then the annotation holds for all classes in this package}{]}.
Overriding \lstinline!access=Access.hide! or \lstinline!access=Access.packageDuplicate!
has no effect.

{[}\emph{It is currently not standardized which result variables are
accessible for plotting. It seems natural to not introduce new flags for
this, but reuse the} \lstinline!Access.XXX! \emph{definition, e.g., for} \lstinline!Access.icon!
\emph{only the variables can be stored in a result file that can also be
inspected in the class, and for} \lstinline!Access.nonPackageText! \emph{all public
and protected variables can be stored in a result file, because all
variables can be inspected in the class.}

\begin{lstlisting}[language=modelica]
package CommercialFluid // Access icon, documentation, diagram
  package Examples // Access icon, documentation, diagram
    model PipeExample // Access everything, can be copied
    end PipeExample;

    package Circuits // Access icon, documentation, diagram
      model ClosedCircuit // Access everything, can be copied
      end ClosedCircuit;
    end Circuits;

    model SecretExample // No access
      annotation(Protection(access=Access.hide));
    end SecretExample;
    annotation(Protection(access=Access.nonPackageDuplicate));
  end Examples;

  package Pipe // Access icon
    model StraightPipe // Access icon
    end StraightPipe;
    annotation(Protection(access=Access.icon));
  end Pipe;

  package Vessels // Access icon, documentation, diagram
    model Tank // Access icon, documentation, diagram, text
    end Tank;
  end Vessels;
  annotation(Protection(access=Access.nonPackageText));
end CommercialFluid;
\end{lstlisting}
{]}

\subsection{Licensing}\doublelabel{licensing}

In this section annotations within the \lstinline!Protection! annotation are
defined to restrict the usage of the encrypted package:

\begin{lstlisting}[language=modelica]
record Protection
  ...
  String features[:]=fill("", 0) "Required license features";
  record License
    String libraryKey;
    String licenseFile="" "Optional, default mapping if empty";
  end License;
end Protection;
\end{lstlisting}
The \lstinline!License! annotation has only an effect on the top of an encrypted
class and is then valid for the whole class hierarchy. {[}\emph{Usually
the licensed class is a package}{]}. The \lstinline!libraryKey! is a secret string
from the library vendor and is the protection mechanism so that a user
cannot generate his/her own authorization file since the \lstinline!libraryKey! is
unknown to him/her.

The \lstinline!features! annotation defines the required license options. If the
features vector has more than one element, then at least a license
feature according to one of the elements must be present. As with the
other annotations, the \lstinline!features! annotation holds for the respective
class and for all classes that are hierarchically on a lower level,
unless further restricted by a corresponding annotation. If no license
according to the \lstinline!features! annotation is provided in the
authorization file, the corresponding classes are not visible and cannot
be used, not even internally in the package.

{[}\emph{Examples:}
\begin{lstlisting}[language=modelica]
// Requires license feature "LicenseOption"
annotation(Protection(features={"LicenseOption"}));

// Requires license features "LicenseOption1" or "LicenseOption2"
annotation(Protection(features={"LicenseOption1", "LicenseOption2"}));

// Requires license features ("LicenseOption1" and "LicenseOption2") or "LicenseOption3"
annotation(Protection(features={"LicenseOption1 LicenseOption2", "LicenseOption3"}));
\end{lstlisting}

{]}

In order that the protected class can be used either a tool specific
license manager, or a license file (called \lstinline!licenseFile!) must be
present. The license file is standardized. It is a Modelica package
without classes that has a \lstinline!Protection! annotation of the following form
which specifies a sequence of target records, which makes it natural to
define start/end dates for different sets of targets individually:

\begin{lstlisting}[language=modelica]
record Authorization
  String licensor="" "Optional string to show information about the licensor";
  String libraryKey "Matching the key in the class. Must be encrypted and not visible";
  License license[:] "Definition of the license options and of the access rights";
end Authorization;

record License
  String licensee ="" "Optional string to show information about the licensee";
  String id[:] "Unique machine identifications, e.g. MAC addresses";
  String features[:] =fill("", 0) "Activated library license features";
  String startDate ="" "Optional start date in UTCformat YYYY-MM-DD";
  String expirationDate="" "Optional expiration date in UTCformat YYYY-MM-DD";
  String operations[:]=fill("",0) "Library usage conditions";
end License;
\end{lstlisting}
The format of the strings used for \lstinline!libraryKey! and \lstinline!id! are not specified
(they are vendor specific). The \lstinline!libraryKey! is a secret of the library
developer. The \lstinline!operations! define the usage conditions and the following
are default names:

\begin{itemize}
\item
  \lstinline!"ExportBinary"! Binary code generated from the Modelica code of the
  library can be included in binaries produced by a simulation
  tool.
\item
  \lstinline!"ExportSource"! Source code generated from the Modelica code of the
  library can be included in sources produced by a simulation tool.
\end{itemize}

Additional tool-specific names can also be used. To protect the
\lstinline!libraryKey! and the target definitions, the authorization file must
be encrypted and must never show the \lstinline!libraryKey!. {[}\emph{All other
information, especially licensor and license should be visible, in order
that the user can get information about the license. It is useful to
include the name of the tool in the authorization file name with which
it was encrypted. Note, it is not useful to store this information in
the annotation, because only the tool that encrypted the Authorization
package can also decrypt it.}{]}

{[}\emph{Example (before encryption):}

\begin{lstlisting}[language=modelica]
// File MyLibrary\package.mo
package MyLibrary
  annotation(Protection(License(libraryKey="15783-A39323-498222-444ckk4ll",
  licenseFile="MyLibraryAuthorization_Tool.mo_lic), ...));
end MyLibrary;

// File MyLibrary\MyLibraryAuthorization_Tool.mo\
// (authorization file before encryption)
package MyLibraryAuthorization_Tool
  annotation(Authorization(
  libraryKey="15783-A39323-498222-444ckk4ll",
  licensor ="Organization A\nRoad, Country",
 license={
  License(licensee="Organization B, Mr.X",
    id ={"lic:1269"}), // tool license number
  License(licensee="Organization C, Mr. Y",
    id ={"lic:511"}, expirationDate="2010-06-30",
   operations={"ExportBinary"}),
  License(licensee="Organization D, Mr. Z",
    id ={"mac:0019d2c9bfe7"}) // MAC address
  }));
end MyLibraryAuthorization_Tool;
\end{lstlisting}
{]}

\section{Annotations for Functions}\doublelabel{annotations-for-functions}

\subsection{Function Derivative Annotations}\doublelabel{function-derivative-annotations}

See \autoref{using-the-derivative-annotation}

\subsection{Inverse Function Annotation}\doublelabel{inverse-function-annotation}

See \autoref{declaring-inverses-of-functions}.

\subsection{External Function Annotations}\doublelabel{external-function-annotations}

See \autoref{annotations-for-external-libraries-and-include-files}.

\section{Annotation Choices for Modifications and Redeclarations}\doublelabel{annotation-choices-for-modifications-and-redeclarations}

See \autoref{annotation-choices-for-suggested-redeclarations-and-modifications}.

\section{Annotation for External Libraries and Include Files}\doublelabel{annotation-for-external-libraries-and-include-files}

See \autoref{annotations-for-external-libraries-and-include-files}.
